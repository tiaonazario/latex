\chapter{INTRODUÇÃO}

A Engenharia é uma área de conhecimento que se distingue pela criação, geração, aperfeiçoamento e emprego de tecnologias com vistas à produção de bens de consumo e de serviços direcionados para suprir as necessidades da sociedade. Com o intuito de atender a demanda que está ocorrendo na sociedade contemporânea, os cursos de graduação estão em fase de transição, modificando suas estruturas político-pedagógicas e buscando se adequarem às tendências de evolução global. Nesse sentido, os cursos de graduação procurando se adaptarem as novas exigências foram em busca de um mecanismo regulatório que pudessem auxiliar na melhoria dos recursos ofertados.

O curso de Engenharia Elétrica através do seu PPC fornecerá ao egresso a possibilidade de direcioná-lo em busca de recursos adequados para que o mesmo possa desenvolver suas aptidões, habilidades e suas capacidades técnicas e profissionais no sentido de estar habilitado a atuar de uma maneira pontual (atuando na sua formação de engenheiro eletricista propriamente dita) ou de uma maneira multidisciplinar. Essa característica multidisciplinar é obtida através das disciplinas optativas e das atividades complementares oferecidas durante o seu percurso acadêmico.

O engenheiro eletricista é o profissional responsável pela geração, pela transmissão e pela distribuição de energia nos setores de hidrelétrica, termoelétrica e subestações. Além disso, é ele que executa tarefas de supervisão, coordenação e orientação aplicada ao campo da eletrônica, eletrotécnica, telecomunicações, e controle e automação. Em relação à eletrônica, as atividades são ligadas à automação e controle, computação, microeletrônica, circuitos integrados, projetos na área de eletrônica de potência e telecomunicações. Já na área da eletrotécnica, o engenheiro eletricista trabalha junto às hidrelétricas e indústrias, em que atua no desenvolvimento de equipamentos.
Vale ressaltar que a existência do curso de Engenharia Elétrica na UFERSA no Campus de Caraúbas beneficiará, não apenas a região do semiárido, mas sim, todo o país, uma vez que os profissionais aqui formados estarão aptos a ingressarem no mercado de trabalho em qualquer região do Brasil  \cite{ref:ufersa-caraubas}.

\section{OBJETIVO}

Formar um grande profissional.
