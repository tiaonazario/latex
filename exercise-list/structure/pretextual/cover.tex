% CAPA---------------------------------------------------------------------------------------------------

% ORIENTAÇÕES GERAIS-------------------------------------------------------------------------------------
% Caso algum dos campos não se aplique ao seu trabalho, como por exemplo,
% se não houve coorientador, apenas deixe vazio.
% Exemplos: 
% \coorientador{}
% \departamento{}

% DADOS DO TRABALHO--------------------------------------------------------------------------------------
\titulo{Título do Trabalho: Subtítulo do Trabalho}
\autor{Nome Completo do Autor}
\local{Caraúbas - RN}
\data{2022}

% NATUREZA DO TRABALHO-----------------------------------------------------------------------------------
% Opções: 
% - Trabalho de Conclusão de Curso (se for Graduação)
% - Dissertação (se for Mestrado)
% - Tese (se for Doutorado)
% - Projeto de Qualificação (se for Mestrado ou Doutorado)
\projeto{Lista de Exercicios}

% DADOS DA INSTITUIÇÃO-----------------------------------------------------------------------------------
% Se a natureza for Trabalho de Conclusão de Curso, coloque o nome do curso de graduação em "programa"
% Formato para o logo da Instituição: \logoinstituicao{<escala>}{<caminho/nome do arquivo>}
\logoinstituicao{0.15}{data/figures/ufersa}
\instituicao{Universidade Federal Rural do Semi-Árido}
\departamento{Departamento de Engenharia Elétrica}
\programa{Bacharelado em Engenharia Elétrica} 
\orientador{Nome do orientador}
